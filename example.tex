%% Font sizes
% Official template has slides with a tiny font size and others with
% quite a big one. To replicate that, we here use 12pt as the document
% base size (using the 12pt option) and the \verysmall command to
% produce the smaller (unreadable) text.
%
% For a more sane balance, use something smaller as the base size
% (e.g. 10pt) and don't use the \verysmall command for slide text.
%
% Remove 't' to vertically align content within slides.
\documentclass[12pt, t]{beamer}


% Default to UTF8 input encoding: plenty of non-ASCII characters
% in Finnish names etc.
\usepackage[utf8]{inputenc}

% Defaults language to English.
\usepackage[english]{babel}

% Used for dummy text only, can be safely removed
% once \lipsum's are removed from body.
\usepackage{lipsum}

\usetheme{newseye}

% Title of your presentation
\title{NewsEye}

% Subtitle of your presentation, optional
\subtitle{A Digital Investigator for Historical Newspapers}

% Name in curly brackets is used on title page
\author{Mark Granroth-Wilding}

% Set to department or similar unit
\institute{Department of Computer Science}

% Date can be anything.
% Comment out or use \today for today's date.
\date{11.6.2019}



\begin{document}


\frame[plain]{\titlepage}



\section{Section 1}

\begin{frame}{Boiler Plate}
	\verysmall
	NewsEye, funded by the European Union's Horizon 2020 research and
	innovation programme, is a research project advancing the state of the
	art and introducing new concepts, methods and tools for digital
	humanities by providing enhanced access to historical newspapers for a
	wide range of users. With the tools and methods created by NewsEye, 
	crucial user groups will be able to investigate views and perspectives
	on historical events and development and, as a consequence, the project
	will \textbf{change the way European digital heritage data is
	(re)searched, accessed, used and analysed.}
\end{frame}



\section{Section 2}

\begin{frame}{Project Outputs}
	\verysmall
	NewsEye will develop a seamlessly integrated armory of tools and
	methods will be created that will improve users' capability to access,
	analyse and use the content in the digital Libraries of historical
	newspapers. NewsEye will thus seek to improve existing tools on
	(amongst others):
	\begin{itemize}
	    \item \textbf{Text Recognition and Article Separation:} 
		NewsEye will essentially address two major obstacles of current
		research projects dealing with historical newspapers: One is the
		fact that in many cases, conventional Optical Character Recognition
		(OCR) does not provide satisfying results. The other is that text
		recognition results are mostly on newspaper page level only
		instead on the appropriate article level.
		\item \textbf{Multilingual and Uncertainty-aware Semantic Text Enrichment:} 
		While named entity recognition (NER) and linking (NEL) are very
		active research areas, their results still are very weak when
		applied to historical data. The main reason is that most of the
		models require linguistic analysis, which is not robust to noisy
		text recognition.
		\item \textbf{Dynamic Text Analysis:} 
		Tools for exploring large sets of historical newspapers are scarce,
		in particular in terms of advanced ability to discover and express
		historical trends, topics and viewpoints suggested by large-scale
		analysis. 
	\end{itemize}
\end{frame}


\begin{frame}{Where can you find us?}
    \begin{itemize}
        \item \uhref{https://www.newseye.eu/}{Website}
		\item \uhref{https://twitter.com/NewsEyeEU}{Twitter}
		\item \uhref{https://www.newseye.eu/blog/}{Blog} posts, \uhref{https://www.newseye.eu/media/}{Podcasts}
		\item Research \uhref{https://www.newseye.eu/publications/research-publications/}{Publications}
		\item \uhref{https://zenodo.org/communities/newseye/?page=1&size=20}{Zenodo} and OpenAIRE
		\item \uhref{https://www.newseye.eu/media/}{Infographics}
		\item \uhref{https://www.newseye.eu/}{Events}
		\item Project \uhref{https://www.newseye.eu/}{News}
    \end{itemize}
\end{frame}



% Subsection and subsubsections are not show on the TOC
\subsection{Subsection 2.1}

\begin{frame}{Demo slides}
    The previous slides replicate exactly those included in the official
	PPT template.
	\vspace{1cm}

	The following slides demonstrate some more things you can
	do with this Beamer template.
\end{frame}


\begin{frame}{An enumerated list}
    \begin{enumerate}
        \item Level 1
        \begin{enumerate}
            \item Level 2
            \begin{enumerate}
                \item Level 3
            \end{enumerate}
        \end{enumerate}
    \end{enumerate}
\end{frame}



\begin{frame}{A bulleted list}
    \begin{itemize}
        \item Level 1
        \begin{itemize}
            \item Level 2
            \begin{itemize}
                \item Level 3
            \end{itemize}
        \end{itemize}
		\item Level 1
		\item Level 1
    \end{itemize}
\end{frame}



\subsubsection{Subsubsection 2.1.1}

\begin{frame}{A slide with lots of content}
    \lipsum[2]
\end{frame}

\begin{frame}[fragile]{A slide with a subtitle}
  \framesubtitle{Concerning the use of subtitles}

  Use the \verb#\framesubtitle# command to add a subtitle
  underneath your frame's main title.

  \vfill

  Note also that we used \verb#fragile# on this slide: otherwise
  the \verb#\verb# command doesn't work.
\end{frame}

% NB: The [c] makes the frame vertically centered even if
% the default on first line of this file is 't'.
\begin{frame}[c]{A frame with blocks}
	% The blocks will be colored accoring to the faculty color scheme
	\begin{block}{Block with math}
		$\mathbf{X} = \{1, 2, 3\}$
	\end{block}
	\begin{block}{Block with text}
		Some text
	\end{block}
\end{frame}



\setbeamertemplate{headline}[plain]

\begin{frame}[fragile]{Frame numbering}
    Frames are numbered in the top right corner.

	If you want to turn numbering off, use:
	\begin{verbatim}
		\setbeamertemplate{headline}[plain]
	\end{verbatim}

	You can do this in the middle of a presentation, as with this frame.

	To turn it back on, use:
	\begin{verbatim}
		\setbeamertemplate{headline}[normal]
	\end{verbatim}
\end{frame}

\end{document}
